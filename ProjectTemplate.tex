% use default report-type
\documentclass[10pt]{report} 

%%% PACKAGES
\usepackage{booktabs}
\usepackage{array}
\usepackage{paralist}
\usepackage{verbatim}
\usepackage{subfig}

\usepackage[yyyymmdd]{datetime}
\renewcommand{\dateseparator}{/}

\usepackage{lastpage}

\usepackage[utf8]{inputenc}
\usepackage{minitoc}
\usepackage{graphicx}

%%% PAGE DIMENSIONS
\usepackage{geometry}
\geometry{a4paper}

%%% HEADERS & FOOTERS
\usepackage{fancyhdr} % This should be set AFTER setting up the page geometry
\pagestyle{fancy} % options: empty , plain , fancy

\setlength\parindent{0pt}

\fancyhf{}
\fancyhead[L]{Project Template}
%\fancyhead[C]{Chapter \thechapter}
\fancyhead[R]{\today}
\renewcommand{\headrulewidth}{0.4pt}
\fancyfoot[L]{Thijs Dregmans}
\fancyfoot[C]{\thepage\ of \pageref{LastPage}}
\fancyfoot[R]{Version 1.1}
\renewcommand{\footrulewidth}{0.4pt}

%%% SECTION TITLE APPEARANCE
\usepackage{sectsty}
\usepackage{tabularx}
\allsectionsfont{\sffamily\mdseries\upshape}

%%% ToC (table of contents) APPEARANCE
\usepackage[nottoc,notlof,notlot]{tocbibind} % Put the bibliography in the ToC
\usepackage[titles,subfigure]{tocloft} % Alter the style of the Table of Contents
\renewcommand{\cftsecfont}{\rmfamily\mdseries\upshape}
\renewcommand{\cftsecpagefont}{\rmfamily\mdseries\upshape} % No bold!


%%% TITLE PAGE
\title{Project Template: A Template for Documenting Projects}
\date{\today}

\author{Thijs Dregmans}

%%% MINI ToCs
\dominitoc[n]


%%% DOCUMENT CONTENTS
\begin{document}
\maketitle

%%% %%% %%% CHAPTER 1 %%% %%%% %%%
\chapter{Introduction}
\thispagestyle{fancy}

Documentation is an important part of project management. The bigger the projects become, the more goes wrong if you lack documentation. It seems a good idea to get a clear picture about what documenting is all about. It is for this reason, that I want to make a template, I can use in this course and further courses. In this document, I will go into details on what the documents should look like. Some documents will always be used, in every project. Others are only necessary for certain projects. For example, an Entity Relationship Diagram is not needed if the PSP (Potentially Shippable Product) does not interact with or uses any database. But every project that works with Scrum, has a Sprint Backlog. Take a look at the Table of Contents to see what documents I included in this guide.

In this new version, I made some important updates. Check the Changelog to see what happened.

\newpage

%%% %%% %%% TOC %%% %%%% %%%
\addtocontents{toc}{\protect\thispagestyle{fancy}}
\tableofcontents

\newpage

%%% %%% %%% CHAPTER 2 %%% %%%% %%%
\chapter{Preparative Documentation}
\thispagestyle{fancy}

In this chapter, several documents that prepare for the project are discussed. These documents are either written at the start of a project or started at the beginning of a project and written througout the project. For example, the Logbook is the first document that is created. Throughout the project, each action, adjustment or contribution is added.

\medskip
\minitoc

\newpage

%%% %%% SECTION 2 %%% %%%%
\section{Stakeholder Analysis}

Even before a Collaboration Agreement is signed, the team does a Stakeholder Analysis. It needs to know who the stakeholders are, because the stakeholders are (partially) involved in the project. This means that before the team and involved stakeholders sign anything, the team needs to know who the stakeholders are.

\bigskip

The purpose of a Stakeholder Analysis is to determine who the stakeholders are, and how big their stake is.

\subsection{Introduction}

Explain what a stakeholder is. Use an introduction to talk about the project and the need for this analysis. What is known at this stage? Who is the client? Is there a vague idea of what the project will become? The Requirements Analysis will point back to this document. Make it easy to do so.

\subsection{Listing Stakeholders}

At first, the team simply lists the stakeholders. Stakeholders can be humans of flesh and blood, or organizations. Here are some possible stakeholders:

\begin{itemize}
	\item Instructor(s)
	\item Client
	\item Team member(s)
	\item Product Owner
	\item Scribe
	\item End Users
	\item Investors
	\item Administrators
	\item Governments
	\item Producers
	\item Suppliers
	\item Customers
	\item Shareholders
	\item Transporters
	\item Management
	\item ...
\end{itemize}

This list is not complete. Add others if necessary. A good Stakeholder Analysis is part of projectmanagement.

\subsection{3 Category Model}

Go through all the listed Stakeholders. Write down who they are, what their stake is. Categorize them with:

\begin{itemize}
	\item Internal or External
	\item Primary or Secondary
	\item Direct or Indirect
\end{itemize}

Here are the definitions.

\begin{itemize}
	\item \textbf{Internal Stakeholder} An Internal Stakeholder is part of the company or team that does the project.
	\item \textbf{External Stakeholder} An External Stakeholder is not part of the company or team that does the project.
	\item \textbf{Primary Stakeholder} An Primary Stakeholder has an high interest in the project.
	\item \textbf{Secondary Stakeholder} An Secondary Stakeholder has a lower interest in the project.
	\item \textbf{Direct Stakeholder} An Direct Stakeholder is involved with day-to-day activities in the project and cares (mainly) about the process.
	\item \textbf{Indirect Stakeholder} An Indirect Stakeholder is not involved with day-to-day activities in the project and cares (mainly) about the results.
\end{itemize}

\subsection{2 Category Model}

In the previous model, 3 Categories were used. There is another way to plot the Stakeholders:

In the 2 Category Model, the Stakeholders are represented by a pin in a board. There are 2 dimensions:

\begin{itemize}
	\item Power
	\item Interest
\end{itemize}

With these two dimensions, the team can plot them into a 2 by 2 matrix:

\medskip
\begin{tabularx}{1\textwidth} { 
  | >{\raggedright\arraybackslash}l
  || >{\raggedright\arraybackslash}l
  | >{\raggedright\arraybackslash}X | }
 \hline
 High power & \textbf{Keep stakeholders satisfied.} & \textbf{Manage stakeholders closely.} \\
 \hline
 Low power & \textbf{Monitor stakeholders.} & \textbf{Keep stakeholders informed.} \\
 \hline
 \hline
 \textbf{X} & Low interest & High interest \\
 \hline
\end{tabularx}
\medskip

The table shows what to do with the Stakeholders. In the next chapter, the team goes into detail.

\subsection{Involving Stakeholders}

To make the project a success, stakeholders need to be actively involved. It is the job of the Stakeholder Ambassador to do this. In this chapter, he/she lines out what steps he/she will take to actively work with the Stakeholders. Stakeholders have to work with the team, at all times. Stakeholders are humans, and thus selfish. If they feel their interests don't align with the project, they might (actively) oppose the project. This should be prevented. The team does not want that stakeholders become the opponents. They need to work with the stakeholders, and vice versa. Write down how and when stakeholders will be involved. Add these dates to the Calender.

\newpage

%%% %%% SECTION 1 %%% %%%
\section{Collaboration Agreement}

At the beginning of the project, before anything happens, the team writes and signs a Collaboration Agreement. In this contract, ground rules for engagement are recorded. Questions like 'How does the team work?' and 'How are agreements recorded?' are answered. The Collaboration Agreement should at least cover the following topics:

\subsection{General Project Information}

In each document, the team provides some General Project Information. This is especially important in the Collaboration Agreement. This document will be signed by all parties involved; meaning all teammembers, clients and others. This section helps readers to understand the context of the document and project. The information can also be put into a `Introduction'.

\subsection{Roles}

Talk about the roles in the project. There are a number of roles, both in the team and outside:

\begin{itemize}
	\item Instructor(s)
	\item Client
	\item Stakeholder
	\item Team member
	\item Scrum Master
	\item Product Owner
	\item Developer(s)
	\item Scribe
	\item Client Ambasador
	\item Stakeholder Ambasador
	\item Hardware Manager
\end{itemize}

The list above is not complete. There may be other roles applicable in the project. Besides, not all listed roles may be needed in the project. Determine what is relevant for the project. Name each involved party, name their role(s) and sumerize the responsibilties.

\subsection{Timetable}

In this section, the team agrees upon a timetable. From when to when is the project? When is the deadline? How many time is invested by the projectteam and the client? Who is responsible for investing enough time? Are there specific days that are used?

\subsection{Communication Channels}

In this section, the team agrees upon a channel for communication. Some agreements are not put into a document. How are these agreements made? How are teammembers informed the team about sickness? How does the team update the client? Clearly define the prefered communication channels. If this is stricly followed, all comunication is in one place. It can be confusing to have mulitiple place where agreements and other messages are shared. Write in this section about the channels that the team will use.

\subsection{Confidentiality \& Exclusivity}

The client want to know if the PSP is his/hers. Who owns the product and all the intellectual property? Write here about the confidentiality and exclusivity of the project.

\subsection{Reporting \& Management}

As a team, collaboration cannot be avoided, nor should it be. To make collaboration easier, the progress is managed by the team with documents. These documents are also important for traceability of the progress for outsiders. (Instructors, clients, ect.) Where are reports stored? Who has access? How are the product backlog changed? Who is responsible for the Product Backlog and other documents? Is the connected to the Roles? How is the code managed? Is Git used? All these questions should be answered in the agreement.

\subsection{Dispute Resolvement}

How are disputes resolved? It depense on who have a dispute. If the team and the client have a dispute, what is the procedure? If teammembers clash, what is the procedure? Layout the procedures in this part of the agreement.

\subsection{Worst Case Scenario's}

Lastly, the team discusses some worst case scenario's. Of course, the team hopes this is never used. Parts of this, will also be put in the Risk Analysis. There can be linked to this document.

\subsubsection{Sickness}

What happens if a teammember becomes sick? Does the project continue? What if the client gets a terminal illness? What if one of the teammembers dies? Is the project continued, is the scope changed, what is the procedure?

\subsubsection{Absence}

What happens if a teammember is abscent for a long time? How is this treated by the team? What is done if the team cannot come into contact with the client or other party?

\subsubsection{Neglect}

What happens if a teammember or client, or even the complete team, neglects the project or his/her responsibilities? What is the procedure?

\bigskip

There are a lot of questions. The more are answered right now, the less headache there will be when/if it actually happens. At the bottom of the document, all parties involved write their name and date of signing. Each party recieves a copy.

\newpage

%%% %%% SECTION 1 %%% %%%%
\section{Logbook}

The Logbook logs who did what when. The Logbook justifies to the client where the team put in hours and what it brougth them. The 'who', 'what' and 'when' can be easily recorded in an Exel-file.

\subsection{Table}

In the Excel-file, there is a table with the following columns:

\begin{itemize}
	\item Date
	\item Time of start
	\item Work done
\end{itemize}

Addiontally, there is a colomn added for each team member. In this colomn, the number of minutes that is spend by this specific teammember is recorded. Each worksession, represents a new row. The current date and time are added. Then the session is started. At the end of the worksession, a description of the work is recorded and how many time is spent. The timespan can be written down the time in Hours or Minutes.

\newpage

%%% %%% SECTION %%% %%%%
\section{Risk Analysis}

Before starting the project, there needs to be an Risk Analysis. In this document, the team analyses the risk assotiated with the project. The risks are divided into two catagories:

\begin{itemize}
	\item General Risks
	\item Project Specific Risks
\end{itemize}

Both categories have their own chapter.

\subsection{General Risks}

The team can put the Risk Analysis in table-format. This way, reader get a quick overview. In the table, the team records the following things:

\begin{itemize}
	\item \textbf{Risk Scenario} Paint a dangerous scenario.
	\item \textbf{Chance} Try to imagine how likely the scenario is. This is rated on a scale of 0 to 10.
	\item \textbf{Impact} Try to imagin the impact of this scenario on the project. This is also rated on a scale of 0 to 10.  
	\item \textbf{Countermeasures} Depending on the Change and Impact, determine whether countermeasures are needed. If a risk has a high chance but a nihil impact, countermeasures may not do anything.
	\item \textbf{Results} Describe how the countermeasures minimized the risk.  
\end{itemize}

Here is a list with risks that could be mentioned:

\begin{itemize}
	\item Death of a teammember
	\item Sickness of a teammember
	\item A teammember stops with the project
	\item A teammember refuses to do an specific task
	\item Communcations with teachers is insufficent
	\item Communcations with teammembers is insufficent
	\item Communcations with client is insufficent
	\item Client is unclear about his/her needs
	\item Teachers are unclear about project goals
	\item Scope is unclear
	\item Scope is changed without consultation of the projectteam
	\item The project doesn't have a clear defined budget
	\item The project goes over budget
	\item The projectteam doesn't have enough skills to complete the project
	\item The progress is not enough/non-existent
	\item Hardware does not arrive in time.
	\item Software licences expire before finishing the project.
\end{itemize}

\subsection{Project Specific Risks}

For Project Specific Risks, the team creates a similair table with Project Specific Risks. An example is that the PSP is endangered because of an failing internet connection. The Project Specific Risks can be privacy or security related. Do not forget to think about continuity and confidentiality. Mention if relevant.

\subsection{SWOT-analysis}

In some cases, it is appropriate to do an SWOT-analysis. SWOT stands for 'Strength, Weakness, Opportunity, Threat'. This model can be applied to all kinds of things, for example the teammembers, the project, the hardware, a specific sensor, ect. 

Research for example the collaboration of teammembers. Four things are looked at:

\begin{itemize}
	\item \textbf{Strength} - It is observed that the team is diverse. This can be a strength, because different people look at things different. Therefore, problems that would otherwise be overlooked can now be eliminated early on. How can this strength be maximized?
	\item \textbf{Weakness} - Because the team is diverse, it is possible that communication is not smooth. Conflicts may arrise. How can this weakness be minimized?
	\item \textbf{Opportunity} - The team opperates in plesant working environment. This is a opportunity for the team, to complete more work. How can this opportunity be used optimally?
	\item \textbf{Threat} - The team is not given enough resources to complete the project. This is an example of an outside threat. How can this threat be avoided, and/or its impact minimized?
\end{itemize}

Note that Strength and Weakness are internal and Opportunity and Threat are external.

\newpage

%%% %%% SECTION 2 %%% %%%%
\section{Scope of the Project}

In this document, the goal of this project and the PSP is defined. It is important to share this document with all stakeholders, so they know what to expect. There are number of things achieved by writing this document:

\begin{itemize}
	\item Stakeholders have a clear view on what the project is set to achieve.
	\item Risks are minimized.
	\item With the scope, a timeline and budget can be establised.
	\item The client cannot change what he/she wants, without consultation of the projectteam.
\end{itemize}

Especially the last is important. It can be disastrous to the relation with the client, if the team doesn't submit what he/she expects, or if the client wants something that the team doesn't provide. By clearly defining the scope, everyone knows what to expect. The Scope is document that is created at the start of the project. During the project it cannot be changed without consultation of all stakeholders. In other words, the client, teammembers and third parties colaborate on this document.	

Together with the Collaboration Agreement, this document is the foundation for the collaboration and communcation between projectteam and client.

\bigskip

The Scope has a number of parts:

\subsection{Goals}

Here the projectteam discuss the project goals, both from the teachers and the client. After conversations with the client, together with the client, one or multiple goals are determined. Not only should the goals of the project be discussed. The client should make clear what the PSP contributes in the real world. It it does not do anything, the project is basically pointless.

\subsection{Resources}

The projectteam has certain resources. Here the resources that are the team have access to are discussed, and which  can use. It may be that the client provides certain resources. By discussing the resources, the team can determine whether it has all the resources it needs, or it lacks certain resources to complete the project.

\subsection{Deliverables}

Here the PSP (Potentially Shippable Product) is determined. It may be hardware or software or a combination. It can also consists of pure research, depending on the product. Deliverables are determined by the client in colaboration with the team. Also the deadline(s) are mentioned.

\subsection{Out of Scope}

It is important to discuss where the project ends. It's easy to say that the project will solve all problems, however that's very unlikely. In what circumstances should the PSP work? What is part of the project, and what is not part of the project? For example, the PSP uses light intensity as an input. The PSP should function by a light intensity with a value less than x. Otherwise, it is out of scope. The scope has an profound impact on the Acceptence Criteria. All these things should be talked about with the client. Inform him/her about the possibilties and advice about what his/her needs are.

\newpage

%%% %%% SECTION 4 %%% %%%%
\section{Planning}

This document is needed to make the timeline clear. Before creating the Planning, the team needs to finish the Scope. In the Scope, the deadline is mentioned. Based on the deadline, and the workload, which is also defined in the Scope, a timeline can be constructed. The timeline can be put into a number of formats; from an Exel-file, a Trello board, a Word-file, ect.

Because of the semi-scrum work method, an complete planning is not needed. But It is easy to put things on a timeline. This way the team gets a clearer picture of how much time there is to complete a certain task. Note that tasks usually take longer than previously thougth. Keep this is mind when planning the tasks.

\newpage

%%% %%% SECTION 5 %%% %%%%
\section{Recieved Documentation from Client}

The team may or may not recieve documentation from the client. If the project is part of an relay project, meaning another group worked on the project before, there must be documentation. If it is not provided, ask for it directly. Firstly, this documentation prevents the team from doing work that is already done by another group. Secondly, it helps learning more about the already-existing techology. It is advisable to make a summary of this documentation.

\newpage


%%% %%% %%% CHAPTER 3 %%% %%%% %%%
\chapter{Functional Research}
\thispagestyle{fancy}

In this chapter, functional research is discussed. The Requirments Analysis are part of this chapter because it is researching what the PSP should be. This is what requirments are. The requirements are the foundation on which the whole project is further build. Based on the requirements, the team does further research, it designs the PSP and tests the PSP. So it is very important to get this part right.

There are, besides the Requirements Analysis, 3 types of research. The difference between them is the method and topic. If we discovering already known facts, we are studying a topic what has been studied before, so we do Literature Research. If we discover new facts, we either ask stakeholders for information, in which case we are doing User Research, or we use another method. This means we are doing Experimental Research.

\medskip
\minitoc

\newpage

%%% %%% SECTION 1 %%% %%%%
\section{Requirements Analysis}

As already discussed, the Requirments form the base of the project. There are a number of chapters that should be included in the Requirments Analysis.

\subsection{Introduction}

In the introduction of this document, the team discusses the Goal of this document and definitions. In the Requirements Analysis, the team uses very specific terminology, which may be unknown by the user. For this, the team provides a detailed list with definitions, like this document.

\subsubsection{Goal}

The team discusses the goal of both this document and the project. The requirements specify the PSP. The PSP should contribute something in the real world. The requirements elaborate in what situations and how the product should function.

\subsubsection{Definitions}

In this part of the introduction, a list of terms and definions is put. For example,

\textbf{Internet - } A network of computers.

\textbf{... - } ...

\subsubsection{Stakeholders}

There are a number of people who will interact with the PSP. For different people, different things are important. Different people have different requirments. In the Stakeholder Analysis, the team discussed the stakeholders. Use this part of the introduction to point back to the stakeholders. Their importance determines what classification the requirement gets.

Defining requirements based on the needs of stakeholders can be tricky for a couple of reasons:

\begin{itemize}
	\item Users do not know what they want, even if they think they do.
	\item Users do not want to commit to written requirments.
	\item Users insist on new requirements, constantly.
	\item Users do not communicate optimally.
	\item Users do not know how the process works.
\end{itemize}

Keep this in mind.

\subsubsection{Constraints}

There are some things about the project that cannot be changed. These things are mentioned under the 'Contraints'. An example is, in a relay project, the work of the past groups. The team cannot change the foundation that previous groups have laid, without talking to the client. This would change the project, so the scope has to be redefined. The constraints will later be used in checking the requirements for attainability. Examples for contraints are time, space, teammembers, expertise, budget, available technology, ect.

\subsection{Research Question}

The client has a problem or issue that the team is assigned to solve. How can this be done? This can be formulated into a question. We can base research on this question. This is called the 'Research Question'. For now, this is part of the Requirements Analysis, but this can easily become a seperate document. Most of the time, the question cannot be answered in one breath. The question can be formulated in multiple sub-questions, that can be researched.

\subsection{List with Requirements}

Under this chapter of the document, the team specifies the requirements. It organizes them into a list. For this step in the document, the team actually writes the requirments down. Each requirement is:

\begin{itemize}
	\item \textbf{Necessary}, to satisfy the need of the client;
	\item \textbf{Verifiable}, to check whether the client's need is indeed satisfied; and
	\item \textbf{Attainable}, so that the client's need can actually be satisfied. 
\end{itemize}

There are a number of different requirements:

\begin{itemize}
	\item \textbf{Business requirments -} These requirements are on the level of the business level, without referencing detailed functions. For example, having a specific process speed.
	\item \textbf{Customer requirments -} These requirements are relevant for the users. They answer the question about usability, like 'where can the system be used?' and 'how long will the system be in use?'
	\item \textbf{Architectural requirments -} Requirements about the architecture of the system. For example, the system must be programmed in C++ because that suits the client.
	\item \textbf{Structural requirments -} Similarly to Architectural requirements, these requirements specify the structure of the PSP. For example, the system must use specify structures like the Design Pattern 'Abstract Factory'.
	\item \textbf{Behavioral requirments -} These requirements say something about how the system should behave.
	\item \textbf{Functional requirments -} These requirements, different from Behavioral requirements, specify what functions must be implemented.
	\item \textbf{Performance requirments -} These requirments specify to what extend, functions must be executed. What quality, quantity, readiness, lifetime, Et cetera, is expected?
	\item \textbf{Derived requirments -} Other requirements may ask for additional requirements. For example, a requirement asking for long range, may result in a design requirment for low weight.
\end{itemize}

The following list has a number for fields the team needs to consider: Functional, Reliablilty, Performance, Maintainability, Interface, Operability, Environment, Safety, Facility, Regulatory, Transportation, Security, Deployment, Privacy, Training, Design constraints and Personnel.

In this chapter, the team only mentions the requirements, put them into a category and give them a number. In the next chapter, it will discusses each requirment in detail.

\subsection{Requirements Detail}

Here the team discusses each requirement in detail. 

\subsubsection{Description}

The team provides a general description of the requirements. Secondly mentions the category that the requirement is part of. Thirdly, it mentions the 3 charicaristics of good requirements:

\paragraph{Necessary}

Why is this requirement necessary? Ask yourself and write down what would/could happen if this requirement is not met and/or defined?

\paragraph{Verifiable}

As the team writes down the requirement, it needs to determine how to check whether this requirement is satisfied. Can it be checked? If so, write down the Acceptance Criteria. If not, this requirement is not correctly formulated. Determining the AC (Acceptance Criteria) will help the team later in the Testplan.

\paragraph{Attainable}

The requirement should be attainable. Checking this is more complicated than it seems. Go back to the Constraints in the introduction. There are a number of contraints, like technology, time and budget. Keep at least the following things in mind:

\begin{itemize}
	\item \textbf{Technology} Does the technology needed exists, or easily build within the project?
	\item \textbf{Time} Does the team have sufficent time to complete this requirement, besides other requirements? Give an estimate.
	\item \textbf{Budget} Does the provided budget by the client enable the team to fulfill this requirement, besides other requirements?
\end{itemize}

If it stays unclear if the requirement is attainable, it shouldn't be a requirement. Make a goal of it instead. Lastly, dont forget that, with all requirements together, it should still be feasible.

\paragraph{Clear}

Lastly, the requirement should be clear. It can help to ask others if the requirements are clear. Each requirement should be expressed in a single thought. The requirement should not be ambiguous. If someone misunderstands the requirement, the requirement is most likely unclear. But a requirement can also be too 'clear': Requirements are not made to talk about how to implement features. It is to specify when features are done. The team should be defining 'what' rather than 'how'.

\subsection{Interaction between requirements}

The requirements together should give a clear picture of what the PSP should 'look like'. Because the requirements are formulated individualy, there can be some problems in the Interaction between the requirements. Requirements should not be contradictory. Describe how this requirement, interacts with together requirements. 

If requirements specify a list of functions the system needs to have, it is generaly better to make a seperate requirement for each item in the list. A list is only appropriate if the items are codependent.

\newpage

%%% %%% SECTION 2 %%% %%%%
\section{Experimental Research}

The team does Experimental Research to "develop, evaluate or communicate a concept, design, problem or solution to make your ideas concrete, to learn whether they work and discover technical limitations and possibilities." It helps to find answers to the questions by doing various experiments. The results determine the answers to the questions. There are two types of experimental research:

\begin{itemize}
	\item \textbf{Comparing} With this type, the team compares two or more potential solutions to see what works best.
	\item \textbf{Value-driven} With this type, the team does a certain experiment to find a specific value, like a maximal temperature.
\end{itemize}

For each type, there is a specific roadmap. These are further discussed under Research Question.

\subsection{Introduction}

After doing this research, the team should end up with an research report. In this report, the team writes down all the steps, so others can retrace these steps and check the conclusion. In the introduction of this report, the team describes the problem. There is a specific problem or question. Tell why this question is best answered with Experimental Research, instead of other types. Tell about how it can help to discover the answer to the problem or question. Both the question of the report and the question of the project. If the research will result in a value, what value(s) are acceptable?

\subsection{Experiment Protocol}

The mainpart of the Experiment is called the 'Protocol'. The Protocol is an preset steps that the team will take when conducting the research.

\subsubsection{Research Question}

The first part of the Experiment Protocol, is the Research Question. What is actually being researched? There is some question it must answer. What is a good research question? A good research question should be:

\begin{itemize}
	\item \textbf{Clear} The question is clear, so that further explanation is not needed.
	\item \textbf{Focused} The question is narrow enough so that it can be answered.
	\item \textbf{Concise} The question is concise, so that the question is formulated in the fewest words possible.
\end{itemize}

There are 2 types of research questions:

\begin{itemize}
	\item Comparing two or more situations.
	\item Searching for values.
\end{itemize}

Name what type of question this is. How does this become clear? Ask what answer(s) the team can expect. Is the answer clear? Is is possible to give a focused answer? Is the answer concise? Write this down.

\subsubsection{Research Setup}

To answer the question, the team has a specific setup in mind. Specify this setup in this chapter. Is there hardware involved? If so, how is this hardware connected? Give a wiring diagram. Is there software involved? if so, what software? What is the version? Under what circumstances is the team doing the test? Specify as much as possible.

\subsubsection{External Factors}

The results will be influence by the input, the system and external factors. For testing the input and/or system, the team wants to minimize the external factors. Find out and list all external factors that could influence the results. Then mention how the team will minimize their impact. It would be best if the team could neutralize the factors. An example of an external factor is the temperature. The temperature influences, for example, the length of an object. Another external factor can be the internet speed. It depends on the factor, whether it can be eliminated or not. At least, try to describe the factor.

\subsubsection{Research Plan}

In this chapter the team gives a specific list, with steps that someone can take, one by one. If someone had the same setup, minimized the external factors, and followed the plan, then he/she should end up with the same results.

\paragraph{Steps to Take}

An example of a list with steps are:

\begin{enumerate}
	\item Set up the hardware and software as described in 'Research Setup'.
	\item Minimize external factors.
	\item Use this query on the database: \textit{CREATE TABLE expResearch (time1 timestamp not null, time2 timestamp not null)}.
	\item Change the API, so that measurement-timestamps are also stored in the table 'expResearch'. 
	\item Wait until the API has written at least 500 records in the newly created table.
\end{enumerate}

\paragraph{Data Recording}

The last part of the Research Protocol is data recording. Here the team lists all data that has to be recorded. There are a number of things the team has to consider:

\begin{itemize}
	\item What data is recorded?
	\item For how long is data recorded?
	\item What is valid data? (Is certain data not recorded because it is not valid?)
	\item How is data recorded? What method is used?
	\item How is data saved? What format is used?
\end{itemize}

An example of a format is an CSV-file, created by a program specifically written for this research. It is important to record the data in the Research Report. If the data takes a lot of space, try to use an attachment, at the end of the report. The team can even try to compress your data, but be sure to use lossless compression.

\subsection{Analysis}

After finishing the Research Protocol, the team executes the protocol. This way, the team ends up with data stored in the format specified in the chapter 'Data Recording'. Here it analyse the data. Depending on the data and format, the team can apply all kinds of analysis. If the team has a CSV-file, it can use Exel to create graphs to easily see patterns in the recorded data.

Write down what analysis the team has done, and what it has found. For example, the result can be that have put the CSV-file into exel and created a graph. The team has found that 95\% of the measurements are inserted within 8,29 seconds.

\subsection{Conclusion}

In the conclusion, the team talks about what it has learned. Recap the introduction. In the introduction, there is written down what value(s) are acceptable. In the Protocol, the team specified a research question. Answer the question. Lastly, talk about what this answer means and what next step the team can make, now that the team knows the answer.

\newpage

%%% %%%% SECTION 3 %%% %%% 
\section{Literature Research}

Literature Research is done to learn from what others have already found out. There is no need to find what others have already found. Their research can simply be used and studied. With 'Literature Research', the Literature is investigated. The same solution may be useful in this project.

\subsection{Introduction}

This research results in a research report. Here all steps are write down, so others can retrace the steps and check the conclusion. In the introduction of this report, the problem is described. There is a specific problem or question. Tell why the team thinks this question is best answered with Literature Research, instead of other types. Tell about how the team can to discover the problem or question.

\subsection{Research Protocol}

With the Research Protocol, it is made clear how the team did the research. Firstly, the team discusses the question that needs an answer. Secondly, the team discusses the theoretical framework. Lastly, the team talks about how the literature analysis came to be.

\subsubsection{Research Question}

The first part of the Research Protocol, is the Research Question. What is the team actually researching? There is some question the team has to answer. What is a good research question? A good research question should be:

\begin{itemize}
	\item \textbf{Clear} The question is clear, so that further explanation is not needed.
	\item \textbf{Focused} The question is narrow enough so that it can be answered.
	\item \textbf{Concise} The question is concise, so that the question is formulated in the fewest words possible?
\end{itemize}

Ask what answer(s) can be expected. Is the answer clear? Is is possible to give a focused answer? Is the answer concise?

\subsubsection{Theoretical Framework}

There are all kinds of fields of study where they have 'literature'. Exploring the theoretical framework, means a couple of things:

\begin{itemize}
	\item Discussing the field. What is it? Who are main contributers?
	\item Explicit assumptions. During the research, assumptions are made. Make them explicit, by talking about them extensivly.
	\item Connect the researcher with the field. Provide theory from the field, so that the researcher may get a better view of what the field is about.
	\item Knowing the limitations. Having a good grip on the theory of the field, helps understanding where it fails. Generalizations fail sometimes. It helps the researcher it it know where, how and most importantly why.
\end{itemize}

\subsubsection{Research Plan}

Before starting the research, the team needs to have a plan. It already has specified a goal, namely the question to be answered. How is it going to get there? This can also be called the 'Methodology'. Discuss here what sources are available, how the team will search for relevant information and how it will evaluate the information. What sources are reliable?

\subsection{Literature Analysis}

The mainpart of the Experiment is called the 'Literature Analysis'. This analysis is an detailed report on what the team learned. This part can easlily be 3 pages long. Specific instructions cannot be provided here, because of how dependent the analysis is on the literature.

\subsection{Conclusion}

In the conclusion, the team talks about what it has learned. Recap the introduction. In the Protocol, there is a specify research question. Answer the question. Lastly, talk about what this answer means and what next step can be taken, now the answer is known.

\newpage

%%% %%%% SECTION 4 %%% %%% 
\section{User Research}

User Research is done to learn about what the end users need and/or want. There are different ways we can find out:

\begin{itemize}
	\item Survey
	\item Interview
	\item Observation
	\item A/B Tests
	\item Usability Tests
	\item Expert Review
\end{itemize}

We will further discuss these in our Research Plan.

\subsection{Introduction}

After doing this research, the team should end up with an research report. Here it writes down all the steps, so others can retrace the steps. Working with users is different from working with devices. Users are not constant, they want and do different things. The team can do a general User Research, to get an idea of how Users wil interact with the PSP, or what question/problem is.

In the introduction of this report, describe the problem, if there is one. Argue why the team thinks this question is best answered with User Research, instead of other types. There no need to have a problem to do User Research. Using the 'Observation' method, the team can even discover problems.

\subsection{Research Protocol}

With the Research Protocol, it is make clear how the team did the research. Firstly, define who the Users are. Secondly, discuss the question that needs an answer. Thirdly, discuss the research plan. How is the research done?

\subsubsection{Define User}

Before the team can do User Research, it needs to define who the users are. It is more complicated than it seems. The PSP can consist of multiple parts. Each part may have different users. Here, the scope is important again. In the scope, the PSP is  defined. Who will use the PSP? These people are the users. Define for each part of the PSP, the user(s). Elaborate why these people are the users.

\subsubsection{Research Question}

Here, there are 2 options: The team can either need an answer to a specific question, or want to see how users interact with the product.

First, it need to define the context in which the PSP will be used. How is the PSP used? In what scenario's should the PSP work? It has no value to research a secenario that will never happen. On what location(s)? Which other systems are involved? What scenario will never happen? These do not need to be simulated in the research.

\subsubsection{Research Plan}

There are a number of ways to do User Research:

\begin{itemize}
	\item \textbf{Survey} Research with a Survey is very common. Create a list with questions and ask potential users to fill in the questionair. The biggest pro is that the team can ask a lot of people at once, using online tools.
	\item \textbf{Interview} Research by Interview is very handy, because the team can ask what the user means directly. If the interviewer misunderstand the answer, he/she can ask for clarfication. The problem is that it is time-consuming and not always objective. The researcher can, consciously or unconsciously, influence the answers, and therefore the results.
	\item \textbf{Observation} With Research by Observation, the researcher does not interact with the user. He/She lets the user interact with the system; watch and learn. This way, he/she check if the user does the right thing, without instructions. This can easily become an interview, which is not bad in and of itself. The team has to determine if this is beneficial or not. Beware of the tendency to ask questions or to help users, while observing.
	\item \textbf{A/B Tests} With A/B Tests, the team creates 2 or more variants of the system. By letting users interact with the system, the team compares the interactions. This way the team can find the optimal system.
	\item \textbf{Usability Tests} With Usability Tests, the team checks how easy the product is to use. Is it useable?
	\item \textbf{Expert Review} Some systems are too complex for users to evaluate. Let an expert review the system, to see if anything goes wrong. For example with an sales system. The salesman would like to do something which is against company policy. Here, you let a compliance tester use the system. He/she is not a user, but an expert. 
\end{itemize}

Different systems ask for different methods. Different questions ask for different methods. Defined the users and questions. Now ask what kind of research is suitable for this system? Elaborate on this.

\subsubsection{Data recording}
Aks what data must be gathered, how this can be done and how the data will be ordered? If speed is prioritzed, an A/B Test can be used to see what system is faster. In this example, the researcher times the users. In other scenario's, he/she want to know what kind of input is desired by the users, so an Survey may be a better options. Think about how data is saved. Automate the process where possible.

\subsection{Result Analysis}

After the team has formulated the plan, it can be executed. After execution, it gathers the results and orders it for analysis. Each method has an different output. In the Planning phase, the team has thought about how it will save the data. What is expected? Is this expectation met? Analyze the results.

\subsection{Conclusion}

In the conclusion, talk about what the team has learned. Recap the introduction. In the Protocol, the team specified a research question. Answer the question. Lastly, talk about what this answer means and what next step can be made, now the answer is known.

\newpage

%%% %%% %%% CHAPTER 4 %%% %%% %%%
\chapter{Scrum Documentation}
\thispagestyle{fancy}

At some point, there is enough Research done. The team knows what the system must be capable of. The team has defined requirments. Now the system needs to be designed and build, so that it fits the requirements.

First, make a Product Backlog. In the Product Backlog, the team discusses the US. US are formulated based on the Requirements. The Product Backlog is based upon the Requirements.

The process is split in multiple so called 'Sprints'. A Sprint has a maximum duration of 2 workweeks. Each Sprint, there is a specific roadmap to  follow:

\begin{enumerate}
	\item Choose what US will be implemented this Sprint. These US form the 'Sprint Backlog'.
	\item Work on the US selected for this Sprint.
	\item Test the US for this Sprint.
	\item Review what US were completed this Sprint. This is called the 'Sprint Review'.
	\item Update the completed US on the Product Backlog.
\end{enumerate}

The Product Backlog does not necessarily has to be in a document, just like the Sprint Backlogs and Sprint Review. You could also look into the use of Azure DevOps.

\medskip
\minitoc
\medskip

\newpage

%%% %%% SECTION 1 %%% %%%
\section{What is Scrum?}

Scrum is form of Agile. To explain what Agile is, the reader must understand what Watefall is. The Waterfall method is the default method: Developers see a problem, work months on a solution and publish it. There is a specified deadline. At that date, the product falls from the sky, like a waterfall. With Agile, the team works with iterations. Instead of deploying at one deadline, the team deploys ever other week. 

\bigskip

With Scrum, there are Sprints. Each Sprint is a worksession of 2 weeks. The team gathers all features that must be build on a Product Backlog. At the start of a Sprint, the team decides what Product Backlog Items are picked to build. The next 2 weeks, the team works to complete these items. At the end, the tasks are done and the features that are worked on, are depolyed. This way, the team can get feedback regularly, i.e. every Sprint. Therefore, the team refines the Product Backlog to match the changing needs and wants of the client.

\bigskip

This chapter discusses the Product Backlog, Sprint Backlog and Sprint Review. 
There are countless books written about Scrum. If more information is needed, it can be found in books, or online.

\newpage

%%% %%% SECTION 1 %%% %%%
\section{Product Backlog}


This document forms the backbone of the Scrum Documentation. In this document, the team lists all User Stories. There is a specific process that is followed that results in this document.

First, we formulate the User Stories. It is important to know that there is a difference between US and Requirments. A US is a function/feature of the PSP that a user finds valuable. By talking with the client and the user, for example through user research, the team finds out what the user finds valuable. All those things will become User Stories.

\subsection{User Stories}

\subsubsection{Formulating User Stories}

User Stories are formulated in a specific way:

\emph{As a ... I want ... so I can ...}

For example:

\emph{As a Administrative User I want to be able to reset passwords of users so I can reset a user's password if he/she lost their password.}

\subsubsection{Types of User Stories}

There are 4 types of User Stories. Each type has their own goal:

\begin{itemize}
	\item \textbf{Features} Features are also known as User Stories. These terms are used as though they are the same, but they are not. Features describe what het PSP should be/have.
	\item \textbf{Bug Fixes} Bug Fixes are also User Stories. This can be confusing, because the User doesn't care about the bugs. On the contrary, the User doesn't want them. Because the team works with Scrum, each task has to be on the Product Backlog. Bug Fixes take time, so it is a User Story. The 'user' here, is the developer.
	\item \textbf{Technical Debt} Some parts of the project are technical, but not directly visible to the user. It is for this reason that they cannot be put into a Feature. Because they cannot be ignored and the problems grow over time, it is 'debt'. It takes time to, for example, build a back-end. The User doesn't see the back-end.
	\item \textbf{Knowledge Acquisition} Similarly to the Technical Debt, there are tasks that take time, and therefore need to be put on the Product Backlog, but are intangible. The difference between Techical Debt and Knowledge Acquisition is building and learning.
\end{itemize}

For the Feature-type, there is already an example. Here are some examples for the other types:

\emph{As a Developer I want fix the PHP-error in 'login.php' on line 34 when a wrong password is entered so I can finish the project.}

\emph{As a Developer I want build the backend of the node.js server so I can finish the project.}

\emph{As a Developer I want learn more about CSS so I can finish the project.}

As you can see, the last part of the User Story is not contributing. Here, this is not important. Only with the Feature-type User Stories, the goal of the US matters.

\subsubsection{Table Format}

Now that the team has formulated User Stories, it can order them and create the Product Backlog. The Product Backlog is 'alive', in the sense that it can change continuously. The Product Backlog is most efficiently stored in an Exel-file. The team can also use a service like MS DevOps. This possibility is further explored in the last chapter of the Scrum Documentation. If the team uses Excel, there is a table like the one below:

\medskip
\begin{tabularx}{0.8\textwidth} { 
  | >{\raggedright\arraybackslash}X 
  | >{\centering\arraybackslash}X 
  | >{\raggedright\arraybackslash}X 
  | >{\raggedright\arraybackslash}X 
  | >{\raggedright\arraybackslash}X 
  | >{\raggedright\arraybackslash}X 
  | >{\raggedleft\arraybackslash}X | }
 \hline
 ID & Priority & Role & Reason & Tasks & AC & Status \\
 \hline
 US001 & M & ... & ...  & ... & ... & Pending \\
 \hline
\end{tabularx}
\medskip

As you can see, there are 6 columns:

\begin{enumerate}
	\item \textbf{ID -} This column numbers all User Stories, so the team can easily reference them. US can change, but the ID should never change. If a US becomes obsolete or the meaning changes significantly, remove the US and add a new one (with a new ID).
	\item \textbf{Priority -} Use the MoSCoW-system to prioritize the work. MoSCoW stands for: Must, Should, Could and Would.
	\item \textbf{Role} The part of the US that shows who the US applies to: As a ...
	\item \textbf{Reason} The part of the US that shows what and why the Role wants: I want ... so I can ...
	\item \textbf{Tasks -} List the task that are asociated with this US here.
	\item \textbf{AC -} Acceptance Criteria. This part is deeply correlated with the Test Report. In the Acceptance Criteria, specify when the Requirement can get the Status 'Finished'. When writing these down, also create the 'Test Report' document. It is important that the team creates the Testplan before designing the system. This way, it will not test what the system can, but if the system satifies the client.
	\item \textbf{Status -} A US has one of 6 states: 
		\emph{Pending}, \emph{Todo}, \emph{Discussing}, \emph{Developing}, \emph{Confirming} or \emph{Finished}
\end{enumerate}

The Product Backlog is a living document that is the foundation for each Sprint. Each Sprint, the Product Backlog is updated and used as a source for new tasks.

\newpage

%%% %%% SECTION 2 %%% %%%
\section{Sprint Backlog}

At the beginning of each Sprint, the team comes together to create the Sprint Backlog. From the Product Backlog, a number of User Stories are selected. The table as shown below, and also shown in the previous chapter.

\subsection{Sprint Planning with US}

The selected US are put in the Sprint Planning. 

\medskip
\begin{tabularx}{0.8\textwidth} {
  | >{\raggedright\arraybackslash}X 
  | >{\centering\arraybackslash}X 
  | >{\raggedright\arraybackslash}X 
  | >{\raggedright\arraybackslash}X 
  | >{\raggedright\arraybackslash}X 
  | >{\raggedright\arraybackslash}X 
  | >{\raggedleft\arraybackslash}X | }
 \hline
 ID & Priority & Role & Reason & Tasks & AC & Status \\
 \hline
 US001 & M & ... & ...  & ... & ... & Pending \\
 \hline
\end{tabularx}
\medskip

The chosen US should be put into the Sprint Backlog. To complete the User Stories, there are tasks writen down. After Listing the User Stories in the table, list all tasks that should be completed in the coming sprint. 

Try to find as much US the team will be able to complete in the next Sprint. Keep the Priority and depencencies in mind. 'Musts' should be implemented before all others. Some User Stories are dependent on others. First, do the independent User Stories, before the dependent, for the simple reason that it is not possible to complete a User Story that is dependent on another unfinished US.

\subsubsection{Tasks}

After choosing the US that the team will work on this Sprint, delegate the workload. Give each teammember an task. Teammembers should have an approximately even workload. The Sprint Backlog is a great place and time to record these arrangements. Write down specifically who is responsible for what. During the Sprint, each day actually, it recommended to organize a Stand Up Meeting. Here the team shortly updates the other teammembers on what they're working on. Here they can ask for feedback and help. This way, teammembers can identify bottlenecks early in the Sprint and prevent wasting of time.

\newpage

%%% %%% SECTION 3 %%% %%%
\section{Sprint Review}

The Sprint Review is very similair to the Sprint Backlog, the difference being that the Backlog plans for a Sprint and the Review looks back on the Sprint.

\subsection{Sprint Review with US}

Repeat the Table, but than updated: 

\medskip
\begin{tabularx}{0.8\textwidth} { 
  | >{\raggedright\arraybackslash}X 
  | >{\centering\arraybackslash}X 
  | >{\raggedright\arraybackslash}X 
  | >{\raggedright\arraybackslash}X 
  | >{\raggedright\arraybackslash}X 
  | >{\raggedright\arraybackslash}X 
  | >{\raggedleft\arraybackslash}X | }
 \hline
 ID & Priority & Role & Reason & Tasks & AC & Status \\
 \hline
 US001 & M & ... & ... & ... & ... & Confirming \\
 \hline
\end{tabularx}
\medskip

Then, give a short description of the tasks that were completed and the tasks that are not. Discuss specifically what is not yet done. Discuss who was responsible for the unfinished tasks and what is being done to prevent not finishing tasks, in the next Sprint.

\newpage

%%% %%% SECTION 5 %%% %%%
\section{Daily Standup}

The Daily Standup meeting is part of Scrum. The team organizes a Daily Standup meeting, every day preferably, at least twice a week. The purpose of this meeting is to inform the team about progress of the workload. Each Sprint, the team distributes the work that needs to be done in the Sprint. Each member is assigned tasks. In the Daily Standup meeting, the team informs on the progress.

If something doesn't work or there are some problems, the team member mentions this to the team. Communication is key. The team can hold individual members responsible for not completing tasks, but at some point the team might needs to start working together on uncompleted tasks. Some people do not have the skills and/or knowledge to complete specific tasks. Work together as a team to fix the problems.

\subsection{Burndown Chart}

This is a chart that is used to show the progress of the team.

During the workshops, we hope to learn more about this!

\newpage

%%% %%% SECTION 5 %%% %%%
\section{Azure DevOps}

Azure DevOps is a Microsoft app, that helps teams work Agile. They have integrated the planning, collaboration, Backlog, Git, CI/CD and pipelines all into the same platform. 

This document can begin to explore the Azure DevOps app, but Microsoft is much better at this. The following courses are recommended:

\begin{itemize}
	\item https://learn.microsoft.com/en-us/training/modules/get-started-with-devops/
	\item https://learn.microsoft.com/en-us/training/paths/evolve-your-devops-practices/
\end{itemize}

\newpage

%%% %%% %%% CHAPTER 5 %%% %%% %%%
\chapter{Design}
\thispagestyle{fancy}

Up until this point, it is possible that the team has no idea what technology to use. It can become a website, desktop-application or fysical device. It is at this point it designs the PSP. It really depends on what the team comes up with, what diagrams/designs it ends up with. If it finds out that an website is the best solution, the project may need a database, and for the database the team makes an ERD. If the team finds that a fysical device is most helpful, it needs to make an Wiring Diagram. Here is a list with some designs/diagrams:

\begin{itemize}
	\item Architecture Diagram
	\item Wiring Diagram
	\item Enity Relationship Diagram
	\item Mechatronic Diagram
	\item Flowchart
	\item Class Diagram
	\item Data Flow Diagram
	\item Pseudo Code
\end{itemize}

There is also a Security Report, that belongs here. In this document, security measures are specified. Write about what measures are taken, what measurements are recommended and how the system is secured.

\newpage

%%% %%% %%% CHAPTER 6 %%% %%% %%%
\chapter{Reflection}
\thispagestyle{fancy}

At each pivotal point, it is helpful to reflect. Teammembers reflect on their choices, either consciously or unconsciously. It is helpful to do it conscious, and to give it words.

In the Test Report the team reflects on the requirements and PSP. In presenations, the teammembers reflect on the PSP and the process. In the Demonstration of Product, they reflect op the PSP. In the Personal Reflection, they reflect on the colaboration of the team, their personal contribution and the process.

\medskip
\minitoc

\newpage

%%% %%% SECTION 1 %%% %%%
\section{Test Report}

In this Report the team uses the requirements to reflect on the PSP. It does this for multiple reasons. First of all, the team wants to know if the PSP does what it should do. Secondly, it wants to demonstrate to the client that the PSP works. 

During testing, the team zooms out. The team first tests very small parts. Then it goes one level bigger, until it has tested the whole PSP.

The team starts with one or more Unit Tests for each requirement. Then the team combines multiple Units and write Component Tests for the Components Tests of the system. 

\bigskip

To make sure that the system works as it should, the team should not only have a System Test. If the System Test fails, it has no idea what went wrong. It is for this reason that the team begins by testing on very small level. During development, it continually tests. Only tests that show that a requirement is done, are written down and put into this report.

At the start of the project, the team could implement the Git automated test functionality. Using Git, it can do Continuous Integration and Continuous Development Tests (CI/CD).

\bigskip

The tests should be written right after the Product Backlog is made. This means that the tests are already clear, before the team is building and developing. This has a big advantage: What is being tested is what the client wants, not what the team has. The results are sincere. It happens a lot that tests are written after the PSP is finished, so that the PSP checks all the boxes. No wonder it 'works' but it really doesn't...

\subsection{Introduction}

The Test Report will have a couple of chapters, beginning with an Introduction. It is good to refresh the memory of the reader: What is the project? What is the PSP? Point back to the Requirments and Product Backlog with User Stories, based upon which this report is written.

Secondly, the tests are numbered. Then, for each test, the following things are elaborated:

\subsection{Test Details}

For each User Story, the team writes one or more relevant tests. Keep in mind, that if the Product Backlog, more specificly the User Stories or their Acceptance Criteria, changes, the Testplan also has to change. 

\subsubsection{Type of Test}
There is a difference between a number of different types of tests:

\begin{itemize}
	\item \textbf{Unit Test} In a Unit test, the team tests a 'Unit'. A Unit is a very small part of the system, at the level of the User Story.
	\item \textbf{Regression Test} With Regression Tests, the team tests Regressively. Once it has tested a feature, it normaly never tesst that feature specifically again. With Regression Tests, the team - or Git with CI/CD - reruns all old tests to check if the system still functions like it should. Almost always, regressive testing is automated. The team can set up a pipline in Git to automate this.
	\item \textbf{Component Test} A system is seperated into multiple Units. A number of Units that work together form a Component. A Component Test tests one Component.
	\item \textbf{System Test} With a System Test, the team tests the whole system.
	\item \textbf{Integration Test} The team has Unit Tests to check if specific parts work. There comes a time when it needs to integrate new, already seperatly tested parts, into the system. It then uses an Integration Tests to check if the system still works if the new part is integrated, and, more important, if the Units work together.
	\item \textbf{Acceptance Test} It looks very much like the System Test, but this is specifically for the client. This test is especially to demonstrate to the client that the PSP works as it should.
\end{itemize}

Discuss what type of test this is.

\subsubsection{Relevance}

For each test, point back to the relevant User Stories, Requirements and Acceptance Criteria. The details of each tests further contain at least the following sections:

\paragraph{Scope}

Here, discuss what type of test this is. The team specifies why this test is needed and what this test tests. Just as important, spend time describing what is not part of the test.

\paragraph{Acceptance Criteria}

Repeat the acceptance Criteria. These are written down in the Product Backlog.

\paragraph{Method}

Lay out a very specific Method of approach, in this part. Someone who doesn't know the project, must be able to follow the steps. Very much like the Experimental Research Report.

\subsubsection{Data to Save}

Depending on the type of test, there is different data that needs to be saved. It is possible to store values in a table or in a CSV-file. Write down what data is to be saved. Under the 'Results' heading, the data is shown.

\subsection{Results}

For each test, there is a Result section. Here, the data is shown, either in table-form or otherwise. The data is compared with the AC. If the AC are met, the test is finished successfully. This can be reported. If the AC are not met, the team gives a detailed report on why. Then, elaborate how this can be fixed. Is it a fundamental design flaw or is there something wrong with one of the components? It is also possible that the test is flawed. This can also be examined here. Is it a good test? It is important to keep in mind that finishing a test successfully is not a goal on its own. The test has a goal itself, namely to check the Requirements. If the test does not accuralty test the requirement, the test is wrong. Only then can the test be changed.

\subsection{Conclusion}

Conclude the Test Report by pointing back to the requirements. If all tests are finished successfully, mention this. If not, elaborate what if futher needed to accomplish this. This may become a new User Story, of the type Bug Fixes.

\newpage

%%% %%% SECTION 2 %%% %%%
\section{Personal Reflection}

Reflection on personal work and roles in the project is important. By doing this, teammembers get a better understanding what they did in the project, and how to improve next time. Here are some questions teammembers can ask themselves. 

\begin{itemize}
	\item How have you used your time?
	\item What goals did I have at the start of the project?
	\item Did I achieve those goals?
	\item Did I underestimate or overestimate my abilities?
	\item How did you contribute to the project?
	\item How did you experience working with your teammates?
	\item What feedback did you recieve from your teammates?
	\item What have you learned?
	\item Are satisfied by your own achievements?
	\item What do you find hard to do?
	\item What could you do to achieve more, that you are not doing?
	\item What did you do that worked great, that you take with you to the next project?
\end{itemize}

Use the answers to create a report.

\bigskip

Team members can make their Personal Reflection more organized with an easy format:

\subsection{Reflect on previous goals}

At some point, team members have to have set some goals to reflect on, to do this. They go back to the goals they have set last project. They go through all goals, and discuss their progress towards them. If targets are hit, name it. If not, what is needed to make this happen?

\subsection{Questions}

Go through the questions, mentioned above. Answer what questions are relevant.

\subsection{New goals}

After answering at least 5 questions in the previous section, new goals can be formulated. If the last targets aren't hit, those goals can be propagated to the new list. Combine existing goals and new goals in a list.

\subsection{Concrete Actionplan}

Just having goals is not enough. How are the proposed targets going to be hit? What is needed to accomplish this? Write for each goal down what concrete actions can be taken to progress towards it.

\newpage

%%% %%% SECTION 3 %%% %%%
\section{Presentations}

A presenation can be structured in an almost, unlimited number of ways. Here, only some examples for common scenario's are described. It is likely that the team will have to give multiple presentation during the project. It is really helpful to have a template for the presentation. This way, the team doesn't have use valueable time of creating the layout of the PowerPoint over and over again. Besides, after a while spectators will notice that the presentations are part of the brand. This looks/is professional. The team can simply create a PowerPoint template at the start of the project and use this again and again.

\subsection{Final Presentation}

In this presentation, the main focus is on the PSP. Of course, the team can give the presentation their own spin, and should be encourage to do so. Consider the following parts:

\begin{enumerate}
	\item Introduction
	\item Talk about the client and the problem.
	\item Talk about what potential solutions were thought of.
	\item Talk about the solution and why this the better option.
	\item Talk about what proboem the PSP solves, and how it works.
	\item Give a live demo.
	\item Show the product backlog and point how features are implemented.
	\item Talk about how certain problems came up during development and how these were handled. (optional)
	\item Talk about the team and the tasks that each member did.
	\item Recap how the PSP helps the client and end with a Q\&A.
\end{enumerate}

Introductions can be given in a number of ways. The simple reseaon is stating the obvious: Who are part of the team and what is the project? An other option is beginning with a scenario. This hooks the spectators for the rest of the presentation. Paint a picture with a question. In the end, presenters can answer the question. It can even stated that, at the end, there are some questions for the audience. Nobody said that an introduction should be boring. This is true for all presentations, by the way.

\subsection{Sprint Review}

In this presentation, the main focus is on the Sprint. Follow the following steps:

\begin{enumerate}
	\item Introduction
	\item Talk about the client and the problem.
	\item Show the product backlog. Show the planning that was made and show what tasks were completed.
	\item Talk about how certain problems came up during development and how these were handled.
	\item Talk about the team and the tasks that each member did.
	\item Show what tasks will be completed in the next sprint.
	\item Recap how the PSP helps the client and end with a Q\&A.
\end{enumerate}

\subsection{Process Presentation}

In this presentation, the main focus is on the Process. Follow the following steps:

\begin{enumerate}
	\item Introduction
	\item Talk about the client and the problem.
	\item Talk about how each member experienced the collaboration with the team.
	\item Talk extensively about how to started the work, what each member did and how.
	\item Show the product backlog. Show the planning that was made and show what tasks were completed.
	\item Talk about how certain problems came up during development and how these were handled.
	\item Talk about the team and the tasks that each member did.
	\item Recap how the PSP helps the client.
\end{enumerate}

For the project, other templates may be relevant. The team can easily develop them.

\newpage

%%% %%% SECTION 4 %%% %%%
\section{Proccessed Feedback}

Feedback is very helpful. There are number of ways that the team recieves feedback. It can organize special moments for users to give feedback in User Research. It can ask for feedback during a Sprint Review or other presentation. The team can also recieve feedback when it isn't looking for it. The team may not be in the 'recieve feedback mode' mentally, so it may be hard to recognize it when it is given.

\medskip

To correctly process feedback, it can be very helpful to organize the process. Gather the feedback continually in a document to be processed at Sprint Reviews. This is a central moment where important decisions are made. This way the feedback can be accesed by the team and, if necessary, implemented in a new User Story. Make a document with the following table:

\medskip
\begin{tabularx}{0.8\textwidth} { 
  | >{\raggedright\arraybackslash}X 
  | >{\raggedright\arraybackslash}X 
  | >{\raggedright\arraybackslash}X 
  | >{\raggedright\arraybackslash}X 
  | >{\raggedright\arraybackslash}X 
  | >{\raggedright\arraybackslash}X | }
 \hline
 ID & Date & Recieved Feedback & Feedback Giver & User Story & Comments \\
 \hline
 FB001 & 01-01-2022 & ... & Client & US001 & ... \\
 \hline
\end{tabularx}
\medskip

Not all Feedback is implemented. Why is it not implemented? The document should show how the team processed the feedback. Show that the feedback is considered. If the feedback is implemented in a User Story, point to the User Story. Note that, if feedback is implemented, requirments may need to change. This may change everything, but can definitely be worth it.

\newpage

%%% %%% SECTION 5 %%% %%%
\section{Demonstration of Product}

The Demo is almost always part of the Final Presentation. Here are some additional tips \& tricks for the demonstration:

\begin{itemize}
	\item Test the Demonstration under the conditions it is given under. For example, the wifi may be different with a lot of people.
	\item Have a backup of the Demonstration in the form of a video.
	\item Keep it short. If it takes 10 steps to show the process, there is no need to show them all. All the audience wants to see, is the end result.
	\item What is the problem? Make this super clear!
	\item Do not make the audience watch you type in a password or other things. Have this automized in some way, because the presenter cannot provide commentary while he/she types. Providing commentary is his/hers job!
	\item Practice on location.
	\item On high end projects, there is a duplicate system.
	\item Never, never, never troubleshoot in real time. If the problem cannot be fixed in 30 seconds, don't enter debug mode. Just go to the video.
\end{itemize}

It goes without saying that a Demo shows the functionality of the PSP. Choose to show all functions, or only the key functions, based on the amount of time. The demo should never be longer than 5 minutes. 2 minutes is more than enough. Keep the techical skills of the audience in mind. Do not tire them with irrelevant details.

\newpage

%%% %%% %%% CHAPTER 7 %%% %%% %%%
\chapter{Collaborative Documentation}
\thispagestyle{fancy}

While working together, create some documents to increase productivivity and to make the process more smooth. The teammembers need to interact with the client, and with other teammembers. There may even be third parties involved. During the meetings, the Scribe takes notes. For the next team, there must be documentation, outlining what the current team did.

\medskip
\minitoc

\newpage

%%% %%% SECTION 2 %%% %%%
\section{Notes of Meetings}

It is customary for meetings to have a so called Scribe. He/she is appointed by the projectteam, and is usually a teammember. In the notes, write down what is talked about. Who made certain statements and agreements? It will work as a short report on what was discussed. If someone did not attend the meeting, and scans the document, he/she should get a clear picture what was discussed. Be sure to keep the document short. It should never be longer than 2 pages, even if the meeting was 4 hours long.

\newpage

% 
% Section 3
%

\section{Transfer Documentation}

If the project is finished, and the team stops working on it, someone might go on with the project and futher develop it. Even if the team knows nothing about this, the team should make this documentation. In fact, the teammembers are making these documents while they're working on the project. It is the product backlog, testreports, among others. In the end, hand over all the documents to the client. He/she may not read it all, but he/she needs to have it, to have full control. 

Besides all documents and the PSP, there is a special folder with Transfer Documentation. Here the team assumes that another team of developers continues with the project. What do these developers need to know?

\subsection{Accompanying Transfer Document}

 First of all, it is a good idea to create a special 'accompanying transfer document'. In this document, the team summarizes what it has done, lists all documents in the Transfer Documentation and gives some recommendations. What documents should make up the Transfer Documentations? At least:

\begin{itemize}
	\item Accompanying Transfer Document
	\item Scope of the Project
	\item Requirements Analysis
	\item All Research Reports (Experimental, Literature and User)
	\item Product Backlog
	\item Sprint Backlogs
	\item Sprint Reviews
	\item All Design Documents
	\item All Test Documents
	\item Demonstration of the PSP
\end{itemize}

It may be a good idea, to mention in the Accompanying Transfer Document, that some documents that were made are not included in the Transfer Documentation. They are included anyway, because they help the potentially new projectteam understand what the current team has done.

\newpage

%%% %%% %%% CHAPTER 8 %%% %%% %%%
\chapter{Conclusion}
\thispagestyle{fancy}

As we have seen in this document, are there a lot of documents that are usally forgotten. If a document is not forgotten, there is missing something. I cannot be sure that I have included every detail about every document, but I believe that following this guide, will help you. It'll help you to see the big picture. A lot of the documents are dependent on eachother. The Scope and Collaboration Agreement are the backbone of the collaboration. Based on the Scope, we build the Requirements Analysis and a Risk Analysis. Based on the Requirements Analysis, we build the Product Backlog. The PSP is build to fullful the User Stories. To Build the PSP, we write Research Reports. We do this in Sprints. Each Sprint we write a Sprint Backlog and Sprint Review, based, again, on the Product Backlog. Then we write Test Reports where we test based on the Requirements. I could go on, but you get the point.

\bigskip

Thijs Dregmans

%\appendix

%%% %%% %%% CHAPTER 9 %%% %%%% %%%
\chapter{Definitions}
\thispagestyle{fancy}

\textbf{AC -} Acceptance Criteria

\textbf{Acceptance Criteria -} A clear statement about when a requirement is met. 

\textbf{CI/CD -} Continuous Integration and Continuous Development. A feature of Git that tests your software automatically, everytime you make a change.

\textbf{Client -} The person/organisation in whose name the team works.

\textbf{Constraint -} Things that cannot be changed in the project. For example, the budget or timeframe.

\textbf{CSV -} Comma Seperated Values. A filetype used to store data.

\textbf{Deliverables -} The endresult of the project. In other words, the Potentially Shippable Product and documentation.

\textbf{Dependencies -} Requirements that depend on the successful implementation of other Requirements.

\textbf{Git -} A framework that is used for version controle of files. A well known version is GitHub.

\textbf{Product Backlog -} A 'living' document with among other, the requirements, Acceptance Criteria and Status.

\textbf{Product Owner -} A person that is placed in the projectteam by the client to represent the client. In smaller projects, this role can be fulfilled by the team.

\textbf{Protocol -} A framework or set of rules for research.

\textbf{PSP -} Potentially Shippable Product. Something that can be used and results from the project. For example, a device that can scan a qr-code.

\textbf{Q\&A -} Questions and Answers.

\textbf{Relay Project -} A Project that follows up on an previous project, often done by others.

\textbf{Requirement -} A criteria for the Potentially Shippable Product. For example, a maximum weight of 2 kg for the device.

\textbf{Scrum -} An agile software development framework.

\textbf{Scope -} What is included and not included in the goal of the project.

\textbf{Scrum Master -} A teammember who is responsible for Scrum Documentation.

\textbf{Sprint -} An iteration of Scrum.

\textbf{Sprint Backlog -} A document that repeats requirements from the Product Backlog, that are selected for the Sprint.

\textbf{Sprint Planning -} The table in the Sprint Backlog.

\textbf{Sprint Review -} A document that reports what was accomplished in the Sprint.

\textbf{Stakeholder -} Someone who holds a stake in the product/project. In other words, someone who is effected by the product/project. For example, the client.

\textbf{Stand Up Meeting -} A meeting in the Scrum framework that occours each day where teammembers inform eachother about their tasks.

\textbf{Status -} A codeword that indicates the progress on a particular requirment.

\textbf{SWOT-analysis -} A type of risk analysis. 'SWOT' stands for Strength, Weakness, Opportunity and Threat.

\textbf{Unit Test -} A Test that tests a small part of the system, at the level of the User Story.

\textbf{US -} User Story. A function or feature of the Potentitally Shippable Product.

% Double check alphabetic order

\newpage

%%% %%% %%% CHAPTER 9 %%% %%% %%%
\chapter{Change Log}
\thispagestyle{fancy}

\medskip
\begin{tabularx}{1\textwidth} { 
  | >{\raggedright\arraybackslash}l
  | >{\raggedright\arraybackslash}l 
  | >{\raggedright\arraybackslash}X | }
 \hline
 Version & Date & Description \\
 \hline
 1.0 & 2023/01/07 & First definite version. \\
 \hline
 1.1 & 2023/02/12 & Corrected language errors, moved Collaboration Agreement to Preparative Documentation, Roles more clearly defined, Personal Reflection expanded, Bibliography added, made small textual adjustments. \\
 \hline
 1.2 & 2023/02/12 & Stakeholder Analysis written, what is Scrum written, DevOps written, Daily Standup written. \\
 \hline
\end{tabularx}
\medskip


\newpage
%%% %%% %%% Bibliography %%% %%% %%%

\begin{thebibliography}{1}
\thispagestyle{fancy}

\bibitem{Stakeholder1}
ActiveCampaign (2023, February) \emph{The 10 Types of Stakeholders That You Meet in Business}:

https://www.activecampaign.com/blog/types-of-stakeholders

\bibitem{Stakeholder2}
Forbes (2023, February) \emph{What Is A Stakeholder Analysis? Everything You Need To Know}:

https://www.forbes.com/advisor/business/what-is-stakeholder-analysis/

\bibitem{Requirements1}
Wikipedia (2022, December) \emph{Requirements Analysis}:

https://en.wikipedia.org/wiki/Requirements\_analysis

\bibitem{Requirements2}
Requirements Experts (1993) \emph{Requirements Analysis}:

https://reqexperts.com/wp-content/uploads/2015/07/writing\_good\_requirements.htm

\bibitem{Scope}
Martins, J. (2022, May 3) \emph{De snelle handleiding voor het bepalen van het projectbereik in 8 stappen}:

https://asana.com/nl/resources/project-scope

\bibitem{Risks}
Indeed Editorial Team (2022, May 31) \emph{10 Common Project Risks (Plus the Steps To Solve Them)}:

https://www.indeed.com/career-advice/career-development/project-risks

\bibitem{Previous Documets}
Dregmans, T. (2022) \emph{Documents for previous projects}

\bibitem{Question}
THE WRITING CENTER (2018, August 8) \emph{How to Write a Research Question}:

https://writingcenter.gmu.edu/writing-resources/research-based-writing/how-to-write-a-research-question

\bibitem{Theoretical Framework}
University of Southern California (2023, January 3) \emph{Research Guides}:

https://libguides.usc.edu/writingguide/theoreticalframework

\bibitem{Product Backlog}
 Raeburn, A. (2022, May 26) \emph{What is a product backlog? (And how to create one)}:

https://asana.com/nl/resources/product-backlog

\bibitem{Status Codewords}
Visual Paradigm (2022) \emph{Status of a user story}:

https://www.visual-paradigm.com/support/documents/vpuserguide/2607/2825/86356\_statusofause.html

\bibitem{DevOps}
Microsoft (2022) \emph{Azure DevOps}:

https://dev.azure.com/

\bibitem{Reflection}
Sutton, J. (2021, March 31) \emph{What Is Sports Psychology? 9 Scientific Theories \& Examples}:

https://positivepsychology.com/introspection-self-reflection/spo

\bibitem{Reflection}
Pearne, S. (2022, June 13) \emph{Details of the joint project}:

https://harperjames.co.uk/article/guide-to-collaboration-agreements/\#section-5

\bibitem{Reflection}
Berkun, S. (2011, October 4) \emph{How to give a perfect demo}:

https://scottberkun.com/2011/how-to-give-a-perfect-demo/

\bibitem{Reflection}
Lucid Content Team (2022) \emph{How to Develop a Stellar Scrum Product Backlog}:

https://www.lucidchart.com/blog/how-to-develop-a-product-backlog-in-agile

\end{thebibliography}

\newpage

\end{document}
